\documentclass[11pt]{extarticle}

\title{Assessing the Performance of Matching Methods in Observational Studies}
\author{Jack Potrykus}
\date{\today}

\usepackage[utf8]{inputenc}

\renewcommand{\familydefault}{\sfdefault}
\setlength\parindent{0pt}

\usepackage{sansmath}
\sansmath

\usepackage{microtype}
\usepackage{amsmath}
\usepackage{minted}
\usepackage{breakcites}
\usepackage{amsfonts}
\usepackage{amssymb}
\usepackage{amsthm}
\usepackage{bm}
\usepackage{caption}
\usepackage{subcaption}
\usepackage{bbm}
\usepackage{enumitem}
\usepackage[margin=2.5cm]{geometry}
\usepackage{fancyhdr}
\usepackage{hyperref}
\usepackage{parskip}
\usepackage{listings}
\usepackage{titling}
\usepackage{graphicx}
\usepackage[dvipsnames]{xcolor}
\usepackage[style=authoryear]{biblatex}
\usepackage{upgreek}
\addbibresource{./bibs/Matching.bib}

\newcommand*\ttvar[1]{\texttt{\expandafter\dottvar\detokenize{#1}\relax}}
\newcommand*\dottvar[1]{\ifx\relax#1\else
  \expandafter\ifx\string_#1\string_\allowbreak\else#1\fi
  \expandafter\dottvar\fi}


\usemintedstyle{manni}

\hypersetup{
    breaklinks=true,
    colorlinks=true,
    citecolor=magenta,
    linkcolor=blue,
    % linkcolor=magenta,
    filecolor=magenta,      
    urlcolor=blue,
    % urlcolor=magenta,
}

% Listings env preferences
\lstset{
    basicstyle=\small\ttfamily,
    columns=flexible,
    breaklines=true
}

% Page styling
\makeatletter
\let\thetitle\@title
\let\theauthor\@author
\let\thedate\@date
% \let\thesubtitle\@subtitle
\makeatother

\pagestyle{fancy}
\renewcommand{\headrulewidth}{0pt}
\lhead{\theauthor}
\rhead{\thetitle}
% \lhead{}
% \rhead{}
\lfoot{{\small{}This document is internal to the University of Chicago Department of Statistics.}}
\cfoot{}
\rfoot{\thepage}
\usepackage{lastpage}

% `answer` box: more customizable
\usepackage{tcolorbox}
\tcbuselibrary{breakable}
\newtcolorbox{solution}[1][]
{
  breakable,
  enhanced,
  sharpish corners,
  boxrule     = 0.25mm,
  boxsep      = 5pt,
  left        = 5pt,
  right       = 5pt,
  top         = 0pt,
  bottom      = 5pt,
  parbox      = false,
  colframe    = ForestGreen!30!black,
  colback     = ForestGreen!5,
  coltitle    = white,
  toptitle    = 0pt,
  bottomtitle = 0pt,
  title       = {\bfseries Solution},
  #1,
}

\allowdisplaybreaks

% Shortcuts for math
\newcommand{\reals}{\ensuremath{\mathbbmss R}}
\renewcommand{\vec}[1]{\ensuremath{\boldsymbol{\mathsfbf{#1}}}}
\newcommand{\mat}[1]{\ensuremath{#1}}
\newcommand{\set}[1]{\ensuremath{\mathcal{#1}}}

% Subtitle! https://tex.stackexchange.com/questions/50182/subtitle-with-the-maketitle-page
\newcommand{\subtitle}[1]{%
  \posttitle{%
    \par\end{center}
    \begin{center}\large#1\end{center}
    \vskip0.5em
  }%
}


\begin{document}


\maketitle{}

\section*{Abstract}
\tableofcontents{}
\newpage{}


\section{Introduction}
\label{sec:Introduction}

In the context of observational data, a bipartite matching problem is minimally defined by:
\begin{itemize}
  \item \mat{X}, an $n \times p$ matrix of feature values;
  \item \vec{z}, an $n$-vector of binary treatment assignments;
  \item $d : \reals^p \times \reals^p \mapsto \reals^+$, a function which computes some ``distance'' or ``cost of matching'' between any two row vectors \vec{x_i} and \vec{x_j} of \mat{X}.
\end{itemize}
% The output 

\set{S}, \vec{b}, \mat{A}, \vec{\upbeta}
test \cite{iacus_multivariate_2011}


\section{Literature Review}

\subsection{Measuring Similarity}
Iacus, King, Porro 2011 - set the scene

\subsubsection{Propensity Scoring}


Rosenbaum and Rubin, 1983
\begin{itemize}
  \item The \emph{propensity score} is most often accredited to \ldots
  \item They offer a more general definition of a score: anything affording conditional independence
  \item Score seeks to replicated randomized trial
  \item Review of properties (particularly: when can we make unbiased estimates of ATE)
  \item Proposed use cases
\end{itemize}

Methods for calculating propensity scores: \cite{garrido_methods_2014}
Note: Blocking-based balance metrics

\paragraph{Extensions}
\begin{itemize}
  \item (Optimal) Caliper width: \cite{austin_optimal_2011}
  \item Prognostic score: \cite{hansen_prognostic_2008}
  \item Miettinen score is the root of the above \cite{miettinen_stratification_1976}
  \item Joint use of Prognostic, Propensity, + Mahalanobis, \cite{leacy_joint_2014}
  \item \cite{imai_causal_2004}
\end{itemize}

\subsubsection{(Coarsened) Exact Matching}

\begin{itemize}
  \item Iacus King Porro, 2011: MIB methods
  \item Iacus King Porro, 2012: Causal inference without balance checking
\end{itemize}

\subsection{Balance Assessment}

Garrido et al, 2014: don't use the outcome in the matching
Garrido et al, 2014: balance in mean does not imply balance in scores

\subsection{Matching Algorithms}

\parencite{rosenbaum_optimal_1989}
\parencite{ho_matchit_2011} (Greedy)
\parencite{khan_efficient_2016} (hungarian)
\parencite{munkres_algorithms_1957} (hungarian)
Hungarian -- Munkres, Khan et al 2016

\section{Python Package}

\section{Experiments}

\subsection{Data Generation}

Papers with data generation:
\begin{itemize}
  \item \cite{austin_optimal_2011}
  \item \cite{stuart_prognostic_2013}
\end{itemize}

\section{Results}

\section{Discussion}

\section{Conclusion}


% NOTE: SEE LINK BELOW
% https://tex.stackexchange.com/questions/8458/making-the-bibliography-appear-in-the-table-of-contents
% Special handling of bibliography awaits!!
\cleardoublepage
\phantomsection
\addcontentsline{toc}{section}{Bibliography}
\printbibliography

\end{document}
